\chapter{Introduction}

As the amount of data generated by individuals and companies exponentially increases over the years, machine learning (ML) methods become very suitable and permit various impactful applications: Cellphones lock systems by facial recognition, automatic diagnosis in medical exams, credit scoring in the banking industry, estimated time arrival in transport companies, or social media ads recommendation.

Although, for statistical purposes, the more data, the better, additional care must be taken when such information is sensitive, such as financial transactions, educational records, genomic data, and socioeconomic attributes. In the last years, regulators have raised concerns about data privacy, releasing and updating laws protecting the rights of data owners, for example, General Data Protection Regulation, GDPR \cite{gdpr}.

This work is an intersection of two areas: Statistics and Cryptography, although they might appear completely unrelated at first sight, such a combination can yield a possible solution for dealing with machine learning applications that have to use sensitive data. A cryptography scheme might empower data privacy since, after encryption, the original data is inaccessible by untrusted parties, But then the data becomes unavailable for executing statistical models too!

A natural question emerges: Can we still compute something on a dataset after encrypting it?  Although we're discussing a contemporaneous, this question was first proposed in a 1978 article \cite{Rivest1978} by the creators of the RSA cryptography system, one of the most famous and important encryption schemes.

In a classical public-key cryptography scheme, there is a secret key $\sk$, kept only by those who can see the original data, and a public key $\pk$ that can be publicly shared and it's used for encryption. Besides that, there are two functions: $\enc(\pk,\cdot)$ for encryption and $\dec(\sk,\cdot)$ for decryption. Suppose we have two messages $m_1,m_2$ (it can be our sensitive data) in a given space and a bivariate function $f(\cdot)$ in this space (an ML training algorithm, for example). The problem proposed by the article is the existence of an alternative function $f'$ in the encrypted domain, the set generated by all possible encrypted messages, such that:
\begin{align*}\dec(\sk,f'(\enc(\pk,m_1),\enc(\pk,m_2))=f(m_1,m_2)
\end{align*}
The above equation says that we can use $f'$ in the encrypted messages, and after decryption, the result would be the same as applying $f$ directly to the plain messages. A cryptography scheme that satisfies this relation is called \textit{Fully Homomorphic Encryption (FHE)}.

The existence of an FHE scheme remained an open problem until 2009, in Gentry's Ph.D. thesis \cite{gentry2009fully}. He proposed a scheme that satisfied the above relation for $f$ being an addition or multiplication function. Although the question was mathematically answered, Gentry's scheme was very inefficient in practice due to the computational cost of encrypted operations and storage space used by encrypted messages.

The subsequent papers after his work focused on proposing better and more practical schemes. A parallel research group has also emerged, studying how to modify training or prediction ML algorithms to the FHE setup, so that they could be executed over a completely encrypted dataset. 

This work has three main chapters: Chapter \ref{ch:algebra} is an abstract algebra review the theory needed for cryptography; Chapter \ref{ch:fhe} goes through FHE research, formally defining Rivest's original problem, explaining Gentry's solution and presenting a more recent and practical scheme, known as HEAAN \cite{ckks17}. Chapter \ref{ch:logit} is an application of HEAAN scheme to the logistic regression training algorithm, with experimental results on a real dataset.



