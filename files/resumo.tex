\setlength{\absparsep}{18pt} 

\begin{resumo}[Abstract]
 \begin{otherlanguage*}{english}
 The amount of data generated by individuals and enterprises is growing exponentially over the last decades, which empowers the use of machine learning methods since, for statistical purposes, the more data a model can have access to, the more accurately it will predict or represent reality. The problem emerges when the model must deal with sensitive data such as medical records, financial history, or genomic data, in which additional care must be taken in order to protect the privacy of data owners. Encrypting sensitive data might appear a good solution at first sight, but it can considerably limit the ability to do statistical analysis. This work is a survey on \textit{Fully Homomorphic Encryption (FHE)}, a special kind of cryptography scheme that still permits some machine learning methods to run over encrypted data, while it has strong mathematical guarantees of privacy protection.  
 \end{otherlanguage*}

 Keywords: Machine Learning, Cryptography, Fully Homomorphic Encryption, Logistic Regression.
\end{resumo}

\begin{resumo}[Resumo]
    A quantidade de dados gerados por pessoas e empresas está crescendo exponencialmente nas últimas décadas, o que potencializa o uso de métodos de aprendizado de máquina, pois, para fins estatísticos, quanto mais dados um modelo tiver acesso, mais preciso ele será na previsão ou representação da realidade. O problema surge quando o modelo precisa lidar com dados sensíveis, como registros médicos, histórico financeiro ou dados genômicos, nos cuidado adicional deve ser tomado para proteger a privacidade dos proprietários dos dados. Criptografar tais dados, pode parecer uma boa solução à primeira vista, mas pode limitar consideravelmente a capacidade de fazer análises estatísticas. Este trabalho é uma pesquisa sobre \textit{Criptografia Completamente Homomórgica} um tipo especial de esquema de criptografia que permite a execução de alguns modelos de aprendizado sobre dados criptografados, enquanto, ao mesmo tempo, mantém fortes garantias matemáticas de proteção de privacidade.
    
    Palavras-chave: Aprendizado de Máquinas, Criptografia, Criptografia Totalmente Homomórfica, Regressão Logística.
\end{resumo}
