\newpage

\begin{dedicatoria}
    \vspace*{\fill}
    %\noindent
    \hfill
    \begin{minipage}{.6\textwidth}
        I dedicate this dissertation to my grandfather Valter (in memoriam) and my great-aunt Varly (in memoriam), who were my inspiration for humility and simplicity and always believed in my studies. I know they are proud of me now!
    \end{minipage}
\end{dedicatoria}
 
\begin{agradecimentos}
	I thank God for giving me the strength to finish this work; my family, who always supported my dream to study math; FGV's Center for the Development of Mathematics and Science (CDMC), who made my dream possible with close assistance and financial support; my advisor Rodrigo Targino, who proposed this work theme and guided me through the whole process; my algebra teacher Luciano Guimarães, who helped me learn the theoretical bases of this work and all my teachers from School of Applied Mathematics (EMAp), who always worked thinking in our learning process.
\end{agradecimentos}

% \begin{epigrafe}
% \vspace*{\fill}

% \begin{flushright}
%     \hspace{7.5cm}
%     \textit{
%         ``One of the pervasive risks that we face in the \\ information age is
%         that even if the amount of \\ knowledge in the world is increasing,
%         the gap \\ between what we know and what we think \\ we know may be
%         widening.''} \\
%         \textit{Nate Silver}
% \end{flushright}
% \end{epigrafe}
